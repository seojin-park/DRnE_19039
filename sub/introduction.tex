\section{서론}

\subsection{연구의 필요성}

현재 경기과학고등학교에는 자동으로 기상 상황을 관측하는 자동기상관측장비(AWS)가 하나 존재한다. 그러나 그 가격이 매우 비싸고 측정한 자료가 csv(comma-seperated values) 형식의 파일로만 저장된다는 단점이 있다. 그렇기에 AWS의 유지보수가 어려우며 일반 학생들이 수집한 자료에 쉽게 접근하지 못한다는 단점이 존재한다. 또한 온습도와 강우 여부 등의 기본적인 기상 정보는 우리의 일상생활에도 큰 영향을 미치기에 자신이 위치한 지역의 기상 상황을 정확히 필요하는 것이 중요하다. 기상청 등에서는 여러 지역의 날씨 정보를 제공하고 있으나, 이러한 날씨 정보는 시나 구 단위로 제공되며, 따라서 국지적인 기상 상황을 파악하기에는 부족한 점이 존재한다. 그렇기에 학생들이 경기과학고등학교가 위치한 수원시 장안구 송죽동의 기상 상황을 쉽게 확인할 수 있도록 하는 것이 필요하다 느껴 본 연구를 진행하게 되었다.

\subsection{연구의 목적}
본 연구에서는 기존의 AWS 장치가 가지고 있던 단점을 보완하는 것을 목적으로 하였다. 상대적으로 값이 저렴하고 프로그램의 수정이 쉬운 초소형 컴퓨터 기판인 라즈베리파이(Raspberry pi)와 각종 센서들로 자동기상관측장비를 제작하여 기상관측장비의 유지 및 보수를 쉽게 할 수 있도록 하였으며, 수집한 데이터는 라즈베리파이의 내부 저장소와 MariaDB 서버 데이터베이스에 모두 전송하였다. 또한 본 연구에서 제작한 AWS 장비를 사용함로써 경기과학고등학교의 기상 상황을 실시간으로 쉽게 확인할 수 있을 것으로 기대된다.