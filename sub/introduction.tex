\section{서론}
수온은 평균적인 기후와 비교해 그 해의 기온이 어떠한지를 나타내는 지표 중 하나이자 그 자체로 기후에 영향을 크게 주기도, 받기도 하는 요소로, 지구 환경을 논할 때 빠트릴 수 없다. 이러한 수온을 관측하는 방법으로는 크게 해양부이를 이용한 직접관측과 인공위성 관측 자료를 이용한 산출법이 있는데, 전자의 경우 구름과 같은 요인을 배제하고 정확한 수온을 산출할 수 있다는 장점이 있으나 넓은 영역의 수온을 계산하지 못한다. 반대로 후자는 넓은 지역에 걸친 해수 온도를 알 수 있으나, 대기로 인한 오차를 계산해야 한다는 단점이 있다. 본 연구에서는 인공위성을 이용한 산출 방식을 채택하여, 집계된 레벨 II 데이터를 레벨III 데이터로 가공하는 것에 초점을 둔다.

%레벨 II 데이터, 레벨 III 데이타는 무엇을 말하는가?

\subsection{연구의 필요성}
연구의 진행 과정에서 인공위성과 그 탑재체, 수집 데이터에 대한 지식을 학습할 수 있으며 데이터 크롤링과 가공법 등을 익힐 수 있다. 해표 온도의 변화 추이를 알아보고, 그 이유에 대한 탐구 등의 후속연구와 연계될 가능성 또한 있다. 

\subsection{선행연구 고찰}
%선행연구에서는 Terra 위성의 MODIS 센서를 이용한 SST (Sea Surface Temperature) 산출 (정주용 외, (2002). Terra/MODIS 해수면온도 산출 및 검증. 대기, 12(3), 596-599), 구름 제거 기법과 구름 영향에 따른 신뢰도 부여 (양성수, 양찬수, 박광순 (2010). TeraScan시스템에서 NOAA/AVHRR 해수면온도 산출시 구름 영향에 따른 신뢰도 부여 기법 : 5월 자료 적용. 한국해양환경에너지학회지, 13(3), 165-173) 등의 연구가 있다. 

선행연구에서는 Terra 위성의 MODIS 센서를 이용한 SST (Sea Surface Temperature) 산출 연구 (정주용 외, 2002), 구름 제거 기법과 구름 영향에 따른 신뢰도 부여에 관한 연구 (양성수, 양찬수, 박광순, 2010), TeraScan 시스템에서 NOAA/AVHRR 해수면온도 산출시 구름 영향에 따른 신뢰도 부여 기법에 관한 연구 등이 있다 \cite{Society2020} \cite{Surface2010} . 


\subsection{연구의 목적}
 본 연구에서는 2012년부터 2019년까지, 8년 동안 Terra/Aqua 위성의 MODIS 센서가 수집한 SST 데이터를 해양위성센터로부터 다운로드하여 레벨 III 데이터로 가공하고, 나아가 그의 경향성을 파악하며 해석하는 것에 목적을 둔다.