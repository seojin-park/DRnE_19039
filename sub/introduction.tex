\section{서론}

\subsection{연구의 필요성}

현재 경기과학고등학교에는 자동으로 기상 상황을 관측하는 자동기상관측장비(AWS, Automatic Weather Station)가 하나 존재한다. 그러나 그 가격이 매우 비싸고 AWS의 유지보수가 어려우며 일반 학생들이 수집한 자료에 쉽게 접근하지 못한다는 단점이 존재한다. 또한 온도, 습도, 강우 여부 등의 기본적인 기상 정보는 우리의 일상생활에도 큰 영향을 미치기에 자신이 위치한 지역의 기상 상황을 정확히 필요하는 것이 중요하다. 기상청에서 동네예보를 하고 있으나,  국지적인 기상 상황을 파악하기에는 부족한 점이 존재한다. 그렇기에 학생들이 경기과학고등학교가 위치한 수원시 장안구 송죽동의 기상 상황을 쉽게 확인할 수 있도록 하는 것이 필요하다고 생각되어 본 연구를 진행하게 되었다.

\subsection{연구의 목적}
본 연구에서는 기존의 AWS 장치가 가지고 있던 단점을 보완하는 것을 목적으로 하였다. 최덕환, 임효혁, 김나영 (2016)의 연구에서는 Mems 센서를 활용하여 소형 AWS를 제작하였으나, 현재까지 라즈베리파이(Raspberry pi)를 사용하여 제작한 자동기상관측장비에 관한 연구는 진행되어 있지 않다.\cite{Ref1} 상대적으로 값이 저렴하고 프로그램의 수정이 쉬운 초소형 컴퓨터 기판인 라즈베리파이와 각종 센서들로 제작하여 AWS의 유지 및 보수를 쉽게 할 수 있도록 하였으며, 수집한 데이터는 라즈베리파이의 내부 저장소에 파일로 저장하고, 동시에 MariaDB 서버 데이터베이스에 전송하였다. 또한 본 연구에서 제작한 AWS 장비를 사용함로써 경기과학고등학교의 기상 상황을 실시간으로 쉽게 확인할 수 있을 것으로 기대된다.