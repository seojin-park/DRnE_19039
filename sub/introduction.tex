\section{서론}

\subsection{연구의 필요성}

\subsection{선행연구 고찰}
정주용 외(2002)는 기존 NASA에서 제공한 MODIS PFSST 알고리즘을 이용하여 정확도를 검증하여 한반도 주변 전 영역에 걸쳐 실제 해수면 온도값에 비해 상당히 낮게 계산되는 문제점이 있다고 판단하고 NOAA 일일 합성해수면 온도값을 실제값으로 가정하고 새로운 MCSST 산출 회귀계수를 구하였다\cite{Modis2020}.

\subsection{연구의 목적}
본 연구에서는 기존의 AWS 장치가 가지고 있던 단점을 보완하는 것을 목적으로 하였다. 최덕환, 임효혁, 김나영 (2016)의 연구에서는 Mems 센서를 활용하여 소형 AWS를 제작하였으나, 현재까지 라즈베리파이(Raspberry pi)를 사용하여 제작한 자동기상관측장비에 관한 연구는 진행되어 있지 않다.\cite{Ref1} 상대적으로 값이 저렴하고 프로그램의 수정이 쉬운 초소형 컴퓨터 기판인 라즈베리파이와 각종 센서들로 제작하여 AWS의 유지 및 보수를 쉽게 할 수 있도록 하였으며, 수집한 데이터는 라즈베리파이의 내부 저장소에 파일로 저장하고, 동시에 MariaDB 서버 데이터베이스에 전송하였다. 또한 본 연구에서 제작한 AWS 장비를 사용함로써 경기과학고등학교의 기상 상황을 실시간으로 쉽게 확인할 수 있을 것으로 기대된다.