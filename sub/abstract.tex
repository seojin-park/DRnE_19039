%\maketitle  % command to print the title page with above variables
\makecover  % command to print the title page with above variables

\setcounter{page}{1}
\renewcommand{\thepage}{\roman{page}}

%----------------------------------------------
%   Table of Contents (자동 작성됨)
%----------------------------------------------
\cleardoublepage
\addcontentsline{toc}{section}{Contents}
\setcounter{secnumdepth}{3} % organisational level that receives a numbers
\setcounter{tocdepth}{3}    % print table of contents for level 3
\baselineskip=2.2em
\tableofcontents


%----------------------------------------------
%     List of Figures/Tables (자동 작성됨)
%----------------------------------------------
\cleardoublepage
\clearpage
\listoffigures	% 그림 목록과 캡션을 출력한다. 만약 논문에 그림이 없다면 이 줄의 맨 앞에 %기호를 넣어서 코멘트 처리한다.

\cleardoublepage
\clearpage
\listoftables  % 표 목록과 캡션을 출력한다. 만약 논문에 표가 없다면 이 줄의 맨 앞에 %기호를 넣어서 코멘트 처리한다.


\cleardoublepage
\clearpage

%---------------------------------------------------------------------
%                  영문 초록을 입력하시오
%---------------------------------------------------------------------
%\begin{abstracts}     %this creates the heading for the abstract page
%	\addcontentsline{toc}{section}{Abstract}  %%% TOC에 표시
%	\noindent{
%			Put your abstract here. Once upon a time, \gshs said : `The first, and the best.'
%	}
%\end{abstracts}

%\cleardoublepage
%\clearpage

\begin{abstractskor}
	\addcontentsline{toc}{section}{초록}  %%% TOC에 표시
	\noindent{
현재 경기과학고에는 자동으로 다양한 기상 정보를 수집하는 자동기상관측장비(AWS)가 존재한다. 그러나 기존의 자동기상관측장비는 그 가격이 비싸고 학생들이 관측 결과에 접근하기 어렵다는 문제가 있었다. 따라서 본 연구에서는 비교적 가격이 저렴한 라즈베리파이(Raspberry Pi)에 각종 센서를 이용하고 센서가 수집한 정보를 Python과 C언어를 통해 정리하여 자동기상관측장비를 제작하였다. 또한 측정한 정보를 라즈베리파이 내부 저장소에 저장하고 이를 MariaDB의 데이터베이스에 전송 후 저장해 장시간 수집한 데이터를 체계적으로 관리할 수 있도록 하였으며, 추후 센서를 보정하고 수집한 기상 정보를 학교 웹사이트로 전송하여 학생들이 기상 정보를 쉽게 파악할 수 있도록 활용할 수 있을 것이다.
	}
\end{abstractskor}






