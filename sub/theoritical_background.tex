\section{이론적 배경}

\subsection{MODIS}
MODIS는 MODerate resolution Imaging Spectrometer의 준말로, 크기 1.0 m X 1.6 m X 1.0 m, 질량 228.7 kg이며, Terra, Aqua 위성의 핵심 탑재체이다. MODIS는 55도의 시야각과 705 km의 고도에서 2330 km의 관측폭으로 하루 한 번 혹은 두 번 같은 지점을 촬영한다. 총 36 개인 각 채널의 해상도는 250 m (채널 1 - 2), 500 m (채널 3 - 7), 1 km (채널 8 - 36)이다. 그 중 SST 관측에 쓰이는 것은 약 3.7 - 4.1µm의 대역폭을 가지고 있는 20, 21, 22, 23 번 채널과 10.8 - 12.3 µm의 31, 32 번 채널이다. 
%어디에서 인용했는지 써야함.


\subsection{Terra/Aqua 위성}
 1999 년 12 월 18 일 발사되어 다음 년도 2 월 24 일부터 자료를 송신한 Terra 위성은 1일에 한 지점을 2번 관측한다. 지구 환경과 기후의 변화를 관측하는 것이 목표인 이 위성은 ASTER (Advanced Spaceborne Thermal Emission and Reflection Radiometer), CERES (Clouds and the Earth's Radiant Energy System), MISR (Multi-angle Imaging SpectroRadiometer), MODIS (Moderate-resolution Imaging Spectroradiometer), MOPITT (Measurements of Pollution in the Troposphere) 로 총 6 가지의 센서들을 탑재하였다. Aqua 위성은 2002 년 5 월 4 일 지표면과 대기 중의 물에 관한 연구를 위하여 발사되었으며, AMSR-E (Advanced Microwave Scanning Radiometer-EOS), MODIS (Moderate Resolution Imaging Spectroradiometer), AMSU-A (Advanced Microwave Sounding Unit), AIRS (Atmospheric Infrared Sounder), AIRS (Atmospheric Infrared Sounder), HSB (Humidity Sounder for Brazil), CERES (Clouds and the Earth's Radiant Energy System) 로 총 6 가지 센서들을 탑재하였으나, 그중 AMSR-E와 HSB가 손상되어 작동을 멈추었고, AMSU-A와 CERES는 일부 고장이 발생하였으나 여전히 작동하고 있다. Terra와 Aqua는 다른 위성 Aura와 함께 EOS (Earth Observing System) 의 일부이다. 


