%%!TeX program = xelatex
%% 부득이하게 pdflatex을 사용해야 할 경우 위의 magic comment를 제거하십시오.

% Initiated by 정민석(2014년도 경기과학고 수학과전문교원)
% Continously being modified by 경기과학고 TeX 사용자협회
% Website : http://gshslatexintro.github.io 

\documentclass{gshs_report}
% 아래의 함수를 사용하면 이미지 파일들을 같은 디렉토리 내에 images 라는 이름을 가진 폴더를 생성한 후, 그 폴더 안에 넣어 사용할 수 있습니다.
% 사용하고자 한다면 주석을 푸십시오.
\graphicspath{{images/}}
% 이곳에 필요한 별도의 패키지들을 적어넣으시오.
%\usepackage{...}
\usepackage{verbatim} % for commment, verbatim environment
\usepackage{spverbatim} % automatic linebreak verbatim environment
%\usepacakge{indentfirst}
\usepackage{tikz}
\usepackage{framed,color}
%\tikzset{
%	image label/.style={
%		every node/.style={
			%fill=black,
			%text=white,
%			font=\sffamily\scriptsize,
%			anchor=south west,
%			xshift=0,
%			yshift=0,
%			at={(0,0)}
%		}
%	}
%}
\usepackage{amsmath}
\usepackage{amsfonts}
\usepackage{amssymb}
\usepackage{float}
\usepackage{graphicx}
\usepackage{tabularx}
\usepackage{multirow}
\usepackage{booktabs}
\usepackage{longtable}
\usepackage{gensymb}
\usepackage{tcolorbox}
\usepackage{wrapfig}
%\usepackage{subcaption}
%\usepackage{floatrow}
%\usepackage{pict2e}
%\usepackage[backend=biber,style=authoryear]{biblatex}
%\usepackage{biblatex}
\usepackage{pgfplots}
\pgfplotsset{
	compat=newest,
	label style={font=\sffamily\scriptsize},
	ticklabel style={font=\sffamily\scriptsize},
	legend style={font=\sffamily\tiny},
	major tick length=0.1cm,
	minor tick length=0.05cm,
	every x tick/.style={black},
}

\usetikzlibrary{shapes}
\usetikzlibrary{plotmarks}
\usepackage{listings}
\usepackage{hologo}
\usepackage{makecell}
\usepackage{color}
\lstset{
	basicstyle=\small\ttfamily,
	columns=flexible,
	breaklines=true
}

\citation
\bibdata

%: ----------------------------------------------------------------------
%:               보고서 정보를 입력하시오
% ----------------------------------------------------------------------
% 아래와 같은 command를 만들면 길이가 긴 용어를 간편하게 사용할 수 있습니다. 단, 이미 지정된 함수명들은 새로운 함수명으로 사용할 수 없습니다.
\newcommand{\gshs}{Gyeonggi Science High School for the Gifted }

\researchtype{심화} % 기초 / 심화
\reporttype{중간} % 중간 / 결과

\title{지구 관측 위성 자료를 이용한 해수면 온도 산출} % 제목 개행 시 \linebreak 사용. \\나 \newline 은 안됨.
\englishtitle{}% 제목 개행 시 \linebreak 사용. \\나 \newline 은 안됨.

\author[1] {박서진} % 제 1 저자명
\email[1]{(tallyfieh1247@gmail.com)} % 제 1 저자 이메일
\author[2] {} % 제 2 저자명
\email[2]{} % 제 2 저자 이메일
\advisor{박기현} % 지도교사명
\advisorEmail{(guitar79@naver.com)} % 지도교사 이메일

%%%%%%%%%%%%%%%%%%%%%%%%%%%%%%%%%%%%%%%%%%%%%%%
%%%% researchtype이 '심화'일 경우에만 나타남 %%%%
\professor{} % 지도교수명
\professorEmail{} % 지도교수 이메일
%%%%%%%%%%%%%%%%%%%%%%%%%%%%%%%%%%%%%%%%%%%%%%%%
\summitdate{2020}{8}{4} % 제출일 (연, 월, 일)
\newtheorem{definition}{정의}
 % usepackage 등의 명령어는 여기에.
\usepackage{cite}
\usepackage{textcomp}
\usepackage{tocloft}
\setlength{\cftbeforesecskip}{0pt}
\setlength{\cftbeforesubsecskip}{0pt}
\setlength{\cftbeforesubsubsecskip}{0pt}

% 본문 시작
\begin{document}

	%표지만들기
	%makecover 함수와 관련하여 "Underfull \hbox (badness 10000) in paragraph" 오류는 무시하십시오. (TeXstudio ver 2.9.4 오류 기준)
	%\makecover
	

	%\baselineskip=2.2em         % line spacing in the paragraph
	%\maketitle  % command to print the title page with above variables
%\makecover  % command to print the title page with above variables

\setcounter{page}{1}
\renewcommand{\thepage}{\roman{page}}

%----------------------------------------------
%   Table of Contents (자동 작성됨)
%----------------------------------------------
\cleardoublepage
\addcontentsline{toc}{section}{Contents}
\setcounter{secnumdepth}{3} % organisational level that receives a numbers
\setcounter{tocdepth}{3}    % print table of contents for level 3
\baselineskip=2.2em
\tableofcontents


%----------------------------------------------
%     List of Figures/Tables (자동 작성됨)
%----------------------------------------------
\cleardoublepage
\clearpage
\listoffigures	% 그림 목록과 캡션을 출력한다. 만약 논문에 그림이 없다면 이 줄의 맨 앞에 %기호를 넣어서 코멘트 처리한다.

\cleardoublepage
\clearpage
\listoftables  % 표 목록과 캡션을 출력한다. 만약 논문에 표가 없다면 이 줄의 맨 앞에 %기호를 넣어서 코멘트 처리한다.


\cleardoublepage
\clearpage

%---------------------------------------------------------------------
%                  영문 초록을 입력하시오
%---------------------------------------------------------------------
%\begin{abstracts}     %this creates the heading for the abstract page
%	\addcontentsline{toc}{section}{Abstract}  %%% TOC에 표시
%	\noindent{
%			Put your abstract here. Once upon a time, \gshs said : `The first, and the best.'
%	}
%\end{abstracts}

%\cleardoublepage
%\clearpage

\begin{abstractskor}
	\addcontentsline{toc}{section}{초록}  %%% TOC에 표시
	\noindent{
본 연구에서는 해양위성센터가 제공하는 MODIS/Terra, MODIS/Aqua 자료를 웹상에서 크롤링하여 그를 가공, 레벨III 데이터를 만드는 과정을 다룬다. Ubuntu 환경에서 프롬프트 명령어를 학습하여 Python 코드를 백그라운드로 실행하고, 그를 통해 다운받은 데이터를 시간의 흐름에 따라 월별, 년도별로 변하는 추이와 편차를 계산하여 그러한 변화가 나타내는 원인에 대하여 고찰하는 것이 최종적 목적이다. 
	}
\end{abstractskor}






 % Abstract
	
	%%%%%%%%%%%%%%%%%%%%%%%%%%%%%%%%%%%%%%%%%%%%%%%%%%%%%%%%%%%
	%%%% Main Document %%%%%%%%%%%%%%%%%%%%%%%%%%%%%%%%%%%%%%%%
	%%%%%%%%%%%%%%%%%%%%%%%%%%%%%%%%%%%%%%%%%%%%%%%%%%%%%%%%%%%
	\cleardoublepage
	\clearpage
	\renewcommand{\thepage}{\arabic{page}}
	\setcounter{page}{1}
	
	%각 장을 아래와 같이 sub 폴더 안에 만들어서 넣으면 편리하다.
	\section{서론}
수온은 평균적인 기후와 비교해 그 해의 기온이 어떠한지를 나타내는 지표 중 하나이자 그 자체로 기후에 영향을 크게 주기도, 받기도 하는 요소로, 지구 환경을 논할 때 빠트릴 수 없다. 이러한 수온을 관측하는 방법으로는 크게 해양부이를 이용한 직접관측과 인공위성 관측 자료를 이용한 산출법이 있는데, 전자의 경우 구름과 같은 요인을 배제하고 정확한 수온을 산출할 수 있다는 장점이 있으나 넓은 영역의 수온을 계산하지 못한다. 반대로 후자는 넓은 지역에 걸친 해수 온도를 알 수 있으나, 대기로 인한 오차를 계산해야 한다는 단점이 있다. 본 연구에서는 인공위성을 이용한 산출 방식을 채택하여, 집계된 레벨 II 데이터를 레벨III 데이터로 가공하는 것에 초점을 둔다.

%레벨 II 데이터, 레벨 III 데이타는 무엇을 말하는가?

\subsection{연구의 필요성}
연구의 진행 과정에서 인공위성과 그 탑재체, 수집 데이터에 대한 지식을 학습할 수 있으며 데이터 크롤링과 가공법 등을 익힐 수 있다. 해표 온도의 변화 추이를 알아보고, 그 이유에 대한 탐구 등의 후속연구와 연계될 가능성 또한 있다. 

\subsection{선행연구 고찰}
%선행연구에서는 Terra 위성의 MODIS 센서를 이용한 SST (Sea Surface Temperature) 산출 (정주용 외, (2002). Terra/MODIS 해수면온도 산출 및 검증. 대기, 12(3), 596-599), 구름 제거 기법과 구름 영향에 따른 신뢰도 부여 (양성수, 양찬수, 박광순 (2010). TeraScan시스템에서 NOAA/AVHRR 해수면온도 산출시 구름 영향에 따른 신뢰도 부여 기법 : 5월 자료 적용. 한국해양환경에너지학회지, 13(3), 165-173) 등의 연구가 있다. 

선행연구에서는 Terra 위성의 MODIS 센서를 이용한 SST (Sea Surface Temperature) 산출 연구 (정주용 외, 2002), 구름 제거 기법과 구름 영향에 따른 신뢰도 부여에 관한 연구 (양성수, 양찬수, 박광순, 2010), TeraScan 시스템에서 NOAA/AVHRR 해수면온도 산출시 구름 영향에 따른 신뢰도 부여 기법에 관한 연구 등이 있다 \cite{Society2020} \cite{Surface2010} . 


\subsection{연구의 목적}
 본 연구에서는 2012년부터 2019년까지, 8년 동안 Terra/Aqua 위성의 MODIS 센서가 수집한 SST 데이터를 해양위성센터로부터 다운로드하여 레벨 III 데이터로 가공하고, 나아가 그의 경향성을 파악하며 해석하는 것에 목적을 둔다. % body1
	\section{이론적 배경}

\subsection{MODIS}
MODIS는 MODerate resolution Imaging Spectrometer의 준말로, 크기 1.0 m X 1.6 m X 1.0 m, 질량 228.7 kg이며, Terra, Aqua 위성의 핵심 탑재체이다. MODIS는 55도의 시야각과 705 km의 고도에서 2330 km의 관측폭으로 하루 한 번 혹은 두 번 같은 지점을 촬영한다. 총 36 개인 각 채널의 해상도는 250 m (채널 1 - 2), 500 m (채널 3 - 7), 1 km (채널 8 - 36)이다. 그 중 SST 관측에 쓰이는 것은 약 3.7 - 4.1µm의 대역폭을 가지고 있는 20, 21, 22, 23 번 채널과 10.8 - 12.3 µm의 31, 32 번 채널이다. 
%어디에서 인용했는지 써야함.


\subsection{Terra/Aqua 위성}
 1999 년 12 월 18 일 발사되어 다음 년도 2 월 24 일부터 자료를 송신한 Terra 위성은 1일에 한 지점을 2번 관측한다. 지구 환경과 기후의 변화를 관측하는 것이 목표인 이 위성은 ASTER (Advanced Spaceborne Thermal Emission and Reflection Radiometer), CERES (Clouds and the Earth's Radiant Energy System), MISR (Multi-angle Imaging SpectroRadiometer), MODIS (Moderate-resolution Imaging Spectroradiometer), MOPITT (Measurements of Pollution in the Troposphere) 로 총 6 가지의 센서들을 탑재하였다. Aqua 위성은 2002 년 5 월 4 일 지표면과 대기 중의 물에 관한 연구를 위하여 발사되었으며, AMSR-E (Advanced Microwave Scanning Radiometer-EOS), MODIS (Moderate Resolution Imaging Spectroradiometer), AMSU-A (Advanced Microwave Sounding Unit), AIRS (Atmospheric Infrared Sounder), AIRS (Atmospheric Infrared Sounder), HSB (Humidity Sounder for Brazil), CERES (Clouds and the Earth's Radiant Energy System) 로 총 6 가지 센서들을 탑재하였으나, 그중 AMSR-E와 HSB가 손상되어 작동을 멈추었고, AMSU-A와 CERES는 일부 고장이 발생하였으나 여전히 작동하고 있다. Terra와 Aqua는 다른 위성 Aura와 함께 EOS (Earth Observing System) 의 일부이다. 


 % 이론적 배경
	\section{연구 과정}

\subsection{접근 가능한 데이터 확인}

해양위성센터, 공공데이터 포털, NASA Ocean Color Web 등에서 인공위성을 통해 수집한 SST 데이터를 다운받을 수 있다는 것을 확인하고, 접근 가능한 데이터를 연도와 센서별로 분류하여 \textrm{Table} \ref{Table:data}에 나타내었다.

\begin{center}
\begin{table}[h]
	\caption{다운 받을 수 있는 데이타}
	\begin{tabular}{c|c|c}
		\hline
		센서명          & 자료 시작 시기     & 자료 종료 시기 (2020. 4. 29. 기준) \\ \hline
		AVHRR        & 2011. 9. 1.  & 2020. 4. 21.               \\ \hline
		MODIS(Aqua)  & 2011. 9. 1.  & 2020. 4. 6.                \\ \hline
		MODIS(Terra) & 2011. 9. 1.  & 2020. 4. 7.                \\ \hline
		VIIRS        & 2016. 6. 17. & 2020. 4. 27.               \\ \hline
	\end{tabular}
	\label{Table:data}	
\end{table}
\end{center}
	
\subsection{데이터 크롤링}
Github에서 웹 크롤링 파일을 다운받고 그를 참고해 해양위성센터에서 받을 수 있는, 2012 년부터 2019 년까지의 8년 동안 Terra/Aqua 위성이 MODIS를 통해 수집한 SST데이터를 크롤링하는 코드를 만들었다. 아래는 소스코드이다. 


\lstset{basicstyle=\scriptsize, tabsize=4, numbers=left, keywordstyle=\color{blue}, commentstyle=\color{magenta}}

\begin{lstlisting}[language=python]
# Based on python2

import urllib.request as urllib
import os

# Example file path : 
# http://222.236.46.45/nfsdb/MODISA/2011/09/01/L2/MYDOCBOX.2011.0901.0413.aqua-1.hdf.zip
# http://222.236.46.45/nfsdb/MODIST/2011/09/01/L2/MODOCBOX.2011.0901.0235.terra-1.hdf.zip
# http://222.236.46.45/nfsdb/MODISA/2015/07/11/L2/MYDOCT.2015.0711.0457.aqua-1.hdf.zip
# http://222.236.46.45/nfsdb/MODIST/2011/09/03/L2/MODOCT.2011.0903.0222.terra-1.hdf.zip
# http://222.236.46.45/nfsdb/MODISA/2012/01/02/L2/MYDOCT.2012.01.02.0510.aqua-1.hdf.zip
#full_url = 'http://222.236.46.45/nfsdb/MODISA/2019/01/01/L2/MYDOCT.2019.0101.0000.aqua-1.hdf.zip'

save_dir_name = '../../downloads/'
if not os.path.exists(save_dir_name):
os.makedirs(save_dir_name)

url1 = 'http://222.236.46.45/nfsdb/MODIST'
url2 = 'terra-1.hdf.zip'

full_urls = []

for Yr in range(2012, 2020) :
	for Mo in range(1, 2) :
		for Da in range(1, 32) :
			for Ho in range(0, 24) :
				for Mi in range(0, 60) :
					full_urls.append("{0}/{1:04d}/{2:02d}/{3:02d}/L2/MODOCT.{1:04d}.{2:02d}{3:02d}.{4:02d}{5:02d}.{6}"\
						.format(url1, Yr, Mo, Da, Ho, Mi, url2))

for full_url in full_urls : 
	filename_el = full_url.split("/")
	filename = filename_el[-1]

	if not os.path.exists(filename) :
		try :
			urllib.urlretrieve(full_url, '{0}{1}'.format(save_dir_name, filename))
			print ('Trying {0}'.format(full_url), '{0}{1}\n'.format(save_dir_name, filename))
		except Exception as e: 
			print('error {0} : {1}\n'.format(e, filename))
	else :
		print ('{0} already exists\n'.format(filename))

\end{lstlisting}

\subsection{Ubuntu 환경}
Chrome Remote Desktop을 이용하여 개인 노트북을 통해 Ubuntu 운영체제의 서버 컴퓨터를 이용할 수 있도록 하였으며, 프롬프트에서 ls, cd 등 명령어를 사용하여 파일 디렉토리를 탐색하는 방법을 학습하였다. 많은 양의 데이터를 다운받아야 하기 때문에 도중에 프롬프트 창을 닫더라도 계속 다운받을 수 있도록 nohup 명령어를 이용하여 백그라운드로 파일을 실행하였다. 



 % 원래 있던 문
	\section{결과 및 토의}

현재 2011년 1월 1일부터 2019년 12월 31일까지의 데이터를 다운받아 서버 컴퓨터에 저장한 상태이다. 
%진짜 이 데이타 다 받아진거 맞아?

SST 데이터를 크롤링할 때에 오류로 인하여 도중에 다운로드가 멈추는 일이 빈번하게 일어났는데, 이러한 오류의 원인을 파악하는 것이 추후 필요할 것으로 예측된다. 
	
\section{앞으로의 계획}

다운받은 데이터 영역을 위도와 경도에 따라 격자를 만들어 구획을 나누고 각 격자별 평균 SST를 구하며, 다시 월별, 년도별로 해당 구획의 온도와 그 평균을 구해 월별로 SST의 변화 추세에 대해 파악하고, 지난 8년간 해표 온도의 변화와 그 주기 또한 알아본다. 나아가 그러한 현상이 나타나는 원인까지도 고찰하는 것이 최종적인 목적이다. 
	% \include{sub/further_study}
	\include{sub/appendix}
	
	\bibliography{bibfile} % 참고문헌
	% BibTeX 코드 쉽게 얻어오는 방법 %
	% Google Scholar 에서 검색한 결과에서 `인용'을 클릭한다.
	% BibTeX 코드를 얻고자 한다면, 하단의 `BibTeX' 을 클릭.
	% 코드가 나온다. Ctrl+A, Ctrl+C로 복사, bibfile에 붙여넣기.


\end{document}
